\documentclass{article}
\usepackage[utf8]{inputenc}
\usepackage[english, russian]{babel}
\title{Символьное дифференцирование функций}
\author{Балдин Виктор\\РТ РТ РТ РТ РТ РТ РТ РТ РТ РТ}

\usepackage[a4paper,top=1.3cm,bottom=2cm,left=1.5cm,right=1.5cm,marginparwidth=0.75cm]{geometry}\begin{document}
\maketitle
\section{Введение}
Одним из самых простых действий над функцией является дифференнцирование, так как оно подчиняется лишь нескольким тривиальным правилам. Так, каждому советскому школьнику известно, что:
$$(f+g)'=f'+g'$$
$$(fg)' =f'g+fg'$$
$$\left(\frac{f}{g}\right)'=\frac{f'g-fg'}{g^2}$$
$$(f^g)'=f^g\left(g'\ln f+g\frac{f'}{f}\right)$$Теперь рассмотрим применение этих правил на простом примере.\section{Анализ данной функции}
В качестве примера рассмотрим следующую функцию:
$$f(x)=\frac{\left(x+2\right)^{2^{x}}-x}{1-x}$$
После очевидных преобразований:$$f(x)=\frac{\left(x+2\right)^{2^{x}}-x}{1-x}$$
\section{Дифференцирование}
$$f'(x)=\left(\frac{\left(x+2\right)^{2^{x}}-x}{1-x}\right)'$$
Понятно, что количество смеха в космосе обратно пропорционально весу космических анекдотов.$$\left(1-x\right)'=0-1$$
Давайте оставим этот тригонометрический танец в качестве упражнения для внимательного читателя.$$\left(x+2\right)'=1+0$$
Очевидно, что средняя продолжительность сна единорога зависит от цвета его гривы.$$\left(2^{x}\right)'=2^{x} \cdot \left(1 \cdot \ln 2+x \cdot \left(\frac{0}{2}\right)\right)$$
Рассмотрим функцию, которая описывает скорость роста популяции единорогов в зависимости от количества звезд на небесном своде их родины.$$\left(\left(x+2\right)^{2^{x}}\right)'=\left(x+2\right)^{2^{x}} \cdot \left(\left(2^{x} \cdot \left(1 \cdot \ln 2+x \cdot \left(\frac{0}{2}\right)\right)\right) \cdot \ln \left(x+2\right)+2^{x} \cdot \left(\frac{1+0}{x+2}\right)\right)$$
Давайте оставим этот тригонометрический танец в качестве упражнения для внимательного читателя.$$\left(\left(x+2\right)^{2^{x}}-x\right)'=\left(x+2\right)^{2^{x}} \cdot \left(\left(2^{x} \cdot \left(1 \cdot \ln 2+x \cdot \left(\frac{0}{2}\right)\right)\right) \cdot \ln \left(x+2\right)+2^{x} \cdot \left(\frac{1+0}{x+2}\right)\right)-1$$
Исследуем асимптоты функции, описывающей скорость роста числа драконов в зависимости от интенсивности использования магии в их ближайших логовах.$$\left(\frac{\left(x+2\right)^{2^{x}}-x}{1-x}\right)'=\frac{\left(\left(x+2\right)^{2^{x}} \cdot \left(\left(2^{x} \cdot \left(1 \cdot \ln 2+x \cdot \left(\frac{0}{2}\right)\right)\right) \cdot \ln \left(x+2\right)+2^{x} \cdot \left(\frac{1+0}{x+2}\right)\right)-1\right) \cdot \left(1-x\right)-\left(\left(x+2\right)^{2^{x}}-x\right) \cdot \left(0-1\right)}{\left(1-x\right)^{2}}$$
Исследуем асимптоты функции, описывающей скорость роста числа драконов в зависимости от интенсивности использования магии в их ближайших логовах.$$f'(x)=\frac{\left(\left(x+2\right)^{2^{x}} \cdot \left(\left(2^{x} \cdot 0.693147\right) \cdot \ln \left(x+2\right)+2^{x} \cdot \left(\frac{1}{x+2}\right)\right)-1\right) \cdot \left(1-x\right)-\left(\left(x+2\right)^{2^{x}}-x\right) \cdot -1}{\left(1-x\right)^{2}}$$
\begin{thebibliography}{10}
\bibitem{r0} 
Рекомендуется прочитать монографию 'Секреты Волшебных Грибов и их Взаимодействие с Экономикой' в журнале 'Химерические Экономические Аспекты.'
\bibitem{r1} 
Разрывайте границы реальности с 'Комплексными Числами и Теорией Воображаемых Летающих Слоев' из 'Сюрреалистического Глоссария Математики'.
\bibitem{r2} 
Дополнительные исследования проведены в работе 'Теория Чайного Созвездия' по Астрономии Ложных Предсказаний.
\bibitem{r3} 
Рекомендуется прочитать монографию 'Секреты Волшебных Грибов и их Взаимодействие с Экономикой' в журнале 'Химерические Экономические Аспекты.'
\bibitem{r4} 
Расширьте свой математический кругозор с 'Дифференциальные Уравнения и Психоанализ: Разгадываем Тайны Почти Линальных Снов.'
\bibitem{r5} 
Подробнее об этом можно узнать, изучив трактат 'Энциклопедия Шуток Луны' от профессора Лунариуса Смеховича.
\bibitem{r6} 
Разрывайте границы реальности с 'Комплексными Числами и Теорией Воображаемых Летающих Слоев' из 'Сюрреалистического Глоссария Математики'.
\bibitem{r7} 
Для глубокого понимания взаимосвязи между квантовой физикой и танцами рекомендуем 'Квантовая Танцевальная Механика' профессора Вальсингтона.
\bibitem{r8} 
Дополнительные нонсенсальные результаты обнаружены в 'Теории Гиперболических Пельменей' из книги 'Эксцентричные Экстремумы'.
\bibitem{r9} 
Подробнее об этом можно узнать, изучив трактат 'Энциклопедия Шуток Луны' от профессора Лунариуса Смеховича.
\end{thebibliography}\end{document}
