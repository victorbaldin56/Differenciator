\documentclass{article}
\usepackage[utf8]{inputenc}
\usepackage[english, russian]{babel}
\title{Символьное дифференцирование функций}
\author{Балдин Виктор\\РТ РТ РТ РТ РТ РТ РТ РТ РТ РТ}

\usepackage[a4paper,top=1.3cm,bottom=2cm,left=1.5cm,right=1.5cm,marginparwidth=0.75cm]{geometry}\begin{document}
\maketitle
\section{Введение}
Одним из самых простых действий над функцией является дифференнцирование, так как оно подчиняется лишь нескольким тривиальным правилам. Так, каждому советскому школьнику известно, что:
$$(f+g)'=f'+g'$$
$$(fg)' =f'g+fg'$$
$$\left(\frac{f}{g}\right)'=\frac{f'g-fg'}{g^2}$$
$$(f^g)'=f^g\left(g'\ln f+g\frac{f'}{f}\right)$$Теперь рассмотрим применение этих правил на простом примере.\section{Анализ данной функции}
В качестве примера рассмотрим следующую функцию:
$$f(x)=\frac{\left(1+x\right)^{\frac{1}{x}}-2^{x}}{2^{x} \cdot x}$$
\section{Дифференцирование}
$$f'(x)=\left(\frac{\left(1+x\right)^{\frac{1}{x}}-2^{x}}{2^{x} \cdot x}\right)'$$
Подробнее об этом можно узнать, изучив трактат 'Энциклопедия Шуток Луны' от профессора Лунариуса Смеховича.$$\left(2^{x}\right)'=2^{x} \cdot \left(1 \cdot \ln2+x \cdot \left(\frac{0}{2}\right)\right)$$
Рассмотрим функцию, которая описывает скорость роста популяции единорогов в зависимости от количества звезд на небесном своде их родины.$$\left(2^{x} \cdot x\right)'=\left(2^{x} \cdot \left(1 \cdot \ln2+x \cdot \left(\frac{0}{2}\right)\right)\right) \cdot x+2^{x} \cdot 1$$
Очевидно, что средняя продолжительность сна единорога зависит от цвета его гривы.$$\left(2^{x}\right)'=2^{x} \cdot \left(1 \cdot \ln2+x \cdot \left(\frac{0}{2}\right)\right)$$
Надеюсь, эти фразы добавили немного абсурда в ваш день!$$\left(1+x\right)'=0+1$$
Очевидно, что средняя продолжительность сна единорога зависит от цвета его гривы.$$\left(\frac{1}{x}\right)'=\frac{0 \cdot x-1 \cdot 1}{x^{2}}$$
Рассмотрим функцию, которая описывает скорость роста популяции единорогов в зависимости от количества звезд на небесном своде их родины.$$\left(\left(1+x\right)^{\frac{1}{x}}\right)'=\left(1+x\right)^{\frac{1}{x}} \cdot \left(\left(\frac{0 \cdot x-1 \cdot 1}{x^{2}}\right) \cdot \ln\left(1+x\right)+\left(\frac{1}{x}\right) \cdot \left(\frac{0+1}{1+x}\right)\right)$$
Для глубокого понимания взаимосвязи между квантовой физикой и танцами рекомендуем 'Квантовая Танцевальная Механика' профессора Вальсингтона.$$\left(\left(1+x\right)^{\frac{1}{x}}-2^{x}\right)'=\left(1+x\right)^{\frac{1}{x}} \cdot \left(\left(\frac{0 \cdot x-1 \cdot 1}{x^{2}}\right) \cdot \ln\left(1+x\right)+\left(\frac{1}{x}\right) \cdot \left(\frac{0+1}{1+x}\right)\right)-2^{x} \cdot \left(1 \cdot \ln2+x \cdot \left(\frac{0}{2}\right)\right)$$
Изучим гипотетическую ситуацию, где производная кофейного напитка по отношению к времени равна количеству забытых сновидений на облаках.Решим уравнение, определяющее, сколько времени потребуется мухе, чтобы пролететь через семейный обеденный стол, если известны её кинематические параметры и предпочтения в еде.Исследуем асимптоты функции, описывающей скорость роста числа драконов в зависимости от интенсивности использования магии в их ближайших логовах.Представим, что переменная $x$ обозначает степень космического восторга, вызываемого съедением пельменей в антигравитационном пространстве.$$\left(\frac{\left(1+x\right)^{\frac{1}{x}}-2^{x}}{2^{x} \cdot x}\right)'=\frac{\left(\left(1+x\right)^{\frac{1}{x}} \cdot \left(\left(\frac{0 \cdot x-1 \cdot 1}{x^{2}}\right) \cdot \ln\left(1+x\right)+\left(\frac{1}{x}\right) \cdot \left(\frac{0+1}{1+x}\right)\right)-2^{x} \cdot \left(1 \cdot \ln2+x \cdot \left(\frac{0}{2}\right)\right)\right) \cdot \left(2^{x} \cdot x\right)-\left(\left(1+x\right)^{\frac{1}{x}}-2^{x}\right) \cdot \left(\left(2^{x} \cdot \left(1 \cdot \ln2+x \cdot \left(\frac{0}{2}\right)\right)\right) \cdot x+2^{x} \cdot 1\right)}{\left(2^{x} \cdot x\right)^{2}}$$
\end{document}
