\documentclass{article}
\usepackage[utf8]{inputenc}
\usepackage[english, russian]{babel}
\title{Символьное дифференцирование функций}
\author{Балдин Виктор\\РТ РТ РТ РТ РТ РТ РТ РТ РТ РТ}

\usepackage[a4paper,top=1.3cm,bottom=2cm,left=1.5cm,right=1.5cm,marginparwidth=0.75cm]{geometry}\begin{document}
\maketitle
\section{Введение}
Одним из самых простых действий над функцией является дифференнцирование, так как оно подчиняется лишь нескольким тривиальным правилам. Так, каждому советскому школьнику известно, что:
$$(f+g)'=f'+g'$$
$$(fg)' =f'g+fg'$$
$$\left(\frac{f}{g}\right)'=\frac{f'g-fg'}{g^2}$$
$$(f^g)'=f^g\left(g'\ln f+g\frac{f'}{f}\right)$$Теперь рассмотрим применение этих правил на простом примере.\section{Анализ данной функции}
В качестве примера рассмотрим следующую функцию:
$$f(x)=\frac{\left(1.000000+x\right)^{\frac{1.000000}{x}}-2.000000^{x}}{2.000000^{x} \cdot x}$$
\section{Дифференцирование}
$$f'(x)=\left(\frac{\left(1.000000+x\right)^{\frac{1.000000}{x}}-2.000000^{x}}{2.000000^{x} \cdot x}\right)'$$
$$\f'(x)=\frac{\left(\left(1.000000+x\right)^{\frac{1.000000}{x}} \cdot \left(\left(\frac{0.000000 \cdot x-1.000000 \cdot 1.000000}{x^{2.000000}}\right) \cdot \ln\left(1.000000+x\right)+\left(\frac{1.000000}{x}\right) \cdot \left(\frac{0.000000+1.000000}{1.000000+x}\right)\right)-2.000000^{x} \cdot \left(1.000000 \cdot \ln2.000000+x \cdot \left(\frac{0.000000}{2.000000}\right)\right)\right) \cdot \left(2.000000^{x} \cdot x\right)-\left(\left(1.000000+x\right)^{\frac{1.000000}{x}}-2.000000^{x}\right) \cdot \left(\left(2.000000^{x} \cdot \left(1.000000 \cdot \ln2.000000+x \cdot \left(\frac{0.000000}{2.000000}\right)\right)\right) \cdot x+2.000000^{x} \cdot 1.000000\right)}{\left(2.000000^{x} \cdot x\right)^{2.000000}}$$
\end{document}
